\documentclass[12pt]{article}
\usepackage{amsmath}
\usepackage{cite}
\usepackage[margin=1in]{geometry}

\title{
    Digital Twin KR3 \\ \large Usecase and Application
}

\author{
    Ahmed Ibrahim Almohamed
}

\begin{document}
\maketitle

\newpage

\section*{Introduction}

Efforts to develop the key technologies that contribute to the 
three main capabilities of mirroring,
shadowing and threading are underway in the DT-driven industries.\cite{jiang2021industrial} \\
Mirroring involves creating an exact virtual replica of a physical object or system in real-time. This capability is crucial for applications such as real-time monitoring and diagnostics, virtual prototyping, and testing of products before actual production. Industries like manufacturing, healthcare, and aerospace benefit greatly from precise replication and analysis to ensure quality and efficiency.
\\Shadowing focuses on continuously updating the digital twin with real-time data from its physical counterpart. This real-time data integration is essential for predictive maintenance, real-time performance optimization, and condition monitoring. Sectors such as energy, utilities, and smart cities utilize shadowing to ensure systems operate efficiently and to predict potential failures before they occur, thus avoiding downtime and reducing maintenance costs.\cite{jiang2021industrial}
\\Threading involves integrating and synchronizing multiple digital twins across different systems and processes. This capability is vital for complex system simulations, supply chain management, and coordinated operations across multiple locations. Industries such as logistics, large-scale manufacturing, and integrated urban infrastructure rely on threading to coordinate various elements seamlessly, ensuring smooth and efficient operations.\cite{jiang2021industrial}
\\ In summary, the principles of mirroring, shadowing and threading are fundamental to the key use cases and applications of digital twin technology. These concepts enable comprehensive and integrated approaches to the monitoring, optimisation and management of complex systems across a wide range of industries.
\newpage

\bibliographystyle{IEEEtran}
\bibliography{ref}
\end{document}