\section{Usecases for DT}

\subsection{System specifications}

The existing system consists of a KuKa robot "KR3-R540" with a KC4 controller as a physical twin, with the team from last semester creating a digital mirror of the system using ROS2 and Gazebo.
The next step in this project is to develop a digital shadow of the system. This will involve connecting the physical twin to the digital model in a unidirectional way.
After that we should be able to go to the next step by implementing a full digital twin or bi-directional communication between the digital model and the physical twin, 
reading and monitoring data from the sensors in real time and sending commands to the physical twin via the digital model. 

\subsection{Usecases}

We can take a step towards potential use cases that we can implement by thinking about Industry 4.0 concepts such as CPS and IoT.
We will also have a look at some use cases in automatic small parts storage with industrial robots and Digital Twins.

\subsubsection{ThuRoboticLab (TRL)}
At THU, several KuKa KR3 R540 robots are utilized in the Robotics Lab, 
serving as laboratory equipment to aid students in their learning. 
The robots are primarily programmed offline, 
meaning that a program is written on a PC without testing, 
then copied to a USB stick, 
which is subsequently plugged into the robot cells for uploading.
While this approach is straightforward, 
it lacks security. For instance, 
there is no mechanism to ensure that the code does not move the joints beyond 
the robot's main limits, potentially leading to defects over time. Additionally, 
the programming language used for these robots is KukaLanguage (KL), 
which is not ideal for generalizing robotics concepts. 
A more versatile language, such as Python or C++, would be more suitable, 
particularly within a ROS2 environment.
Moreover, this offline programming method, 
or simply using the robot HMI for control, consumes a significant amount of time 
that could be better spent practicing more concepts.
To address these issues, 
we propose creating a virtual environment to run a digital twin of the robots. 
This would allow students to use a more accessible programming language to test 
their exercises. This environment will be connected to the real robot via OPC UA, 
linking the digital and physical robots. Additionally, 
visualizations reflecting the real data of the robot will enable students to 
verify the outcomes of their runs effectively.

\subsubsection{Optimized Order Picking and Packing}

In an Automatic Small Parts Warehouse (AKL), 
the efficiency and accuracy of order picking and packing are 
crucial for operational success. 
The current method of programming industrial robots using 
KukaLanguage (KL) is not ideal for generalizing robotics 
concepts and is time-consuming. To address these limitations, 
we propose utilizing a digital twin (DT) of the warehouse robots.
The digital twin will allow for the simulation and optimization 
of picking paths and packing strategies. 
By visualizing real-time data, operators can identify the most 
efficient routes and methods for order fulfillment. 
This not only reduces the time required for programming and 
testing but also enhances the accuracy of the picking and 
packing process.
Furthermore, the digital twin will enable operators to 
verify the outcomes of their runs through detailed 
visualizations, ensuring that orders are picked and packed 
correctly.
This approach will lead to more efficient order fulfillment, 
reduced shipping costs, and improved overall warehouse 
operations.
\subsubsection{Real-time Inventory Management}

In an Automatic Small Parts Warehouse (AKL), 
maintaining accurate inventory levels is essential for 
smooth operations. 
The current system's limitations in real-time data collection 
and processing can lead to discrepancies and inefficiencies. 
To overcome these challenges, we propose implementing a digital 
twin (DT) for real-time inventory management. 
The DT will continuously gather data from sensors and 
RFID tags on inventory items, 
updating the virtual model with current stock levels and 
locations. This real-time visualization will enable operators 
to monitor inventory accurately, 
identify low-stock items promptly, and optimize restocking 
processes. By simulating different restocking strategies, 
the DT will help in selecting the most efficient approach, 
ensuring that inventory levels are maintained without 
overstocking or stockouts. This integration will lead to 
improved inventory accuracy, efficient restocking, and overall 
enhanced warehouse management.

\subsubsection{Energy Efficiency and Dynamic Reconfiguration}

Energy consumption is a significant concern in warehouse 
operations, impacting both costs and environmental sustainability. 
The current system lacks the capability to dynamically adjust 
operations for optimal energy use. To address this, we propose 
using a digital twin (DT) to enhance energy efficiency and 
enable dynamic reconfiguration of warehouse processes. 
The DT will simulate various operational scenarios, identifying 
energy-saving opportunities and optimizing robot movements and 
tasks to minimize energy consumption. By continuously monitoring 
energy usage and adjusting operations in real-time, the DT will 
ensure that the warehouse operates at peak efficiency. 
Additionally, the DT will facilitate dynamic reconfiguration of 
workflows based on real-time data, adapting to changes in demand 
and operational conditions. This approach will result in 
significant energy savings, reduced operational costs, and a 
more sustainable warehouse environment.

\subsubsection{Remote Monitoring and Control}

In an Automatic Small Parts Warehouse (AKL), 
the ability to monitor and control operations remotely is 
crucial for maintaining efficiency and addressing issues 
promptly. The current system's reliance on on-site presence 
for monitoring and control limits flexibility and 
responsiveness. To enhance remote capabilities, 
we propose implementing a digital twin (DT) for remote 
monitoring and control. The DT will provide a real-time 
virtual representation of the warehouse, 
accessible from any location. Operators can monitor the 
status of robots, inventory levels, and overall warehouse 
operations through the DT, receiving alerts for any anomalies 
or issues. Additionally, the DT will enable remote control of 
robots and other systems, allowing operators to make adjustments 
and resolve problems without being physically present. 
This remote capability will improve operational flexibility, 
reduce downtime, and ensure that the warehouse can be managed 
efficiently from anywhere.

\subsubsection{Predictive Maintenance}

In an Automatic Small Parts Warehouse (AKL), 
equipment downtime can significantly impact operational 
efficiency and lead to increased costs. 
Traditional maintenance approaches, such as reactive or 
scheduled maintenance, often result in either unexpected 
failures or unnecessary maintenance activities. 
To address these challenges, we propose implementing a digital 
twin (DT) for predictive maintenance.
The digital twin will continuously collect and analyze data 
from various sensors embedded in the warehouse equipment, 
such as robots, conveyors, and storage systems. 
By leveraging advanced analytics and machine learning algorithms,
the DT will predict potential equipment failures before 
they occur. This predictive capability will enable operators to
perform maintenance activities only when necessary, 
based on the actual condition of the equipment.
Furthermore, the DT will provide detailed insights into the 
health and performance of the equipment, 
allowing operators to identify patterns and trends that may 
indicate underlying issues. 
This proactive approach to maintenance will minimize unplanned 
downtime, extend the lifespan of the equipment, 
and reduce maintenance costs.
By integrating predictive maintenance into the warehouse 
operations, the digital twin will ensure that the equipment 
remains in optimal condition, leading to improved reliability, 
efficiency, and overall operational performance.