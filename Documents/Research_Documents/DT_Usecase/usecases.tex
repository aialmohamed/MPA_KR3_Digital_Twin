\section{Usecases for DT}

\subsection{System specifications}

The existing system consists of a KuKa robot "KR3-R540" with a KC4 controller as a physical twin, with the team from last semester creating a digital mirror of the system using ROS2 and Gazebo.
The next step in this project is to develop a digital shadow of the system. This will involve connecting the physical twin to the digital model in a unidirectional way.
After that we should be able to go to the next step by implementing a full digital twin or bi-directional communication between the digital model and the physical twin, 
reading and monitoring data from the sensors in real time and sending commands to the physical twin via the digital model. 

\subsection{Usecases}

We can take a step towards potential use cases that we can implement by thinking about Industry 4.0 concepts such as CPS and IoT.

\subsubsection{ThuRoboticLab (TRL)}
At THU, several KuKa KR3 R540 robots are utilized in the Robotics Lab, 
serving as laboratory equipment to aid students in their learning. 
The robots are primarily programmed offline, 
meaning that a program is written on a PC without testing, 
then copied to a USB stick, 
which is subsequently plugged into the robot cells for uploading.
While this approach is straightforward, 
it lacks security. For instance, 
there is no mechanism to ensure that the code does not move the joints beyond 
the robot's main limits, potentially leading to defects over time. Additionally, 
the programming language used for these robots is KukaLanguage (KL), 
which is not ideal for generalizing robotics concepts. 
A more versatile language, such as Python or C++, would be more suitable, 
particularly within a ROS2 environment.
Moreover, this offline programming method, 
or simply using the robot HMI for control, consumes a significant amount of time 
that could be better spent practicing more concepts.
To address these issues, 
we propose creating a virtual environment to run a digital twin of the robots. 
This would allow students to use a more accessible programming language to test 
their exercises. This environment will be connected to the real robot via OPC UA, 
linking the digital and physical robots. Additionally, 
visualizations reflecting the real data of the robot will enable students to 
verify the outcomes of their runs effectively.