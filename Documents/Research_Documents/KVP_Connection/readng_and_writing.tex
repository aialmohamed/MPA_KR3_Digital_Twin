\section{Reading and Writing}

\subsection{Messages}

In the KUKAVARPROXY protocol the main reading and writing is done to the global variables of the KUKA controller (in this case KRC4).
This means that we read or write to the global variables of the KUKA controller.
According to \cite{openkuka_system_variables} is a list of all the global variables that we are able to control, together with their descriptions.
In the appendix of this file are the tables of all the global variables with the data types and their description.

\subsection{Reading}
Reading message format:
\begin{itemize}
    \item \textbf{2 bytes}: Id (uint16)
    \item \textbf{2 bytes}: Content length (uint16)
    \item \textbf{1 byte}: Read/Write mode (0=Read)
    \item \textbf{2 bytes}: Variable name length (uint16)
    \item \textbf{N bytes}: Variable name to be read (ASCII)
\end{itemize}
To access a variable, 
the client must include two parameters 
in the message: the desired function 
type and the variable name. 
When reading a specific variable, 
the function type should be indicated 
by the character “0”. For example, 
if the client wishes to read the system 
variable \texttt{\$OV\_PRO}, which 
controls the robot's speed override, 
the message sent to the server will 
follow the format illustrated in 
following table\cite{sanfilippo2014jopenshowvar}.

\begin{table}[h!]
    \centering
    \caption{Reading variables}
    \begin{tabular}{|c|c|}
    \hline
    \textbf{Field} & \textbf{Description} \\ \hline
    00 & message ID \\ \hline
    09 & length of the next segment \\ \hline
    0  & type of desired function \\ \hline
    07 & length of the next segment \\ \hline
    \texttt{\$OV\_PRO} & Variable to be read \\ \hline
    \end{tabular}
    \end{table}
In detail, the first two characters of this string represent the 
message identifier (ID), which is a sequential integer ranging 
from 00 to 99. The server's response will include the same ID, 
allowing each request to be matched with its corresponding response, 
even if there is a delay. The next two characters in the reading message 
indicate the length of the following segment in hexadecimal units. 
In this particular instance, 09 accounts for one character denoting the 
function type, two characters specifying the length of the subsequent 
segment, and seven characters for the variable length. 
The fifth character, 0, indicates the type of function requested—in this 
case, a reading operation. Following this, two more characters denote 
the variable length (in hexadecimal units), and finally, the last part
of the message contains the variable itself.\cite{sanfilippo2014jopenshowvar}
\newpage
\subsection{Writing}
\begin{itemize}
    \item \textbf{2 bytes:} Id (uint16)
    \item \textbf{2 bytes:} Content length (uint16)
    \item \textbf{1 byte:} Read/Write mode (1=Write)
    \item \textbf{2 bytes:} Variable name length (uint16)
    \item \textbf{N bytes:} Variable name to be written (ASCII)
    \item \textbf{2 bytes:} Variable value length (uint16)
    \item \textbf{M bytes:} Variable value to be written (ASCII)
\end{itemize}
To write a specific variable, 
the message must include three key 
parameters: the type of function, 
the name of the target variable, 
and the value to be assigned. 
The function type for writing is 
identified by the character “1”. 
For example, if the variable to be 
written is the system variable 
\texttt{\$OV\_PRO}, and the value 
to be assigned is 50 
(indicating a 50\% override speed), 
the client will send a message to the 
server in the format illustrated in the following table\cite{sanfilippo2014jopenshowvar}.
\begin{table}[h!]
    \centering
    \caption{Writing variables}
    \begin{tabular}{|c|c|}
    \hline
    \textbf{Field} & \textbf{Description} \\ \hline
    00 & message ID \\ \hline
    0b & length of the next segment \\ \hline
    1  & type of desired function \\ \hline
    09 & length of the next segment \\ \hline
    \texttt{\$OV\_PRO} & Variable to be written \\ \hline
    50 & value to be written \\ \hline
    \end{tabular}
    \end{table}

